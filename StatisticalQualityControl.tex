
\documentclass[11pt]{article} % use larger type; default would be 10pt

\usepackage[utf8]{inputenc} % set input encoding (not needed with XeLaTeX)

\usepackage{geometry} % to change the page dimensions
\geometry{a4paper}
\usepackage{graphicx} 
\usepackage{booktabs} % for much better looking tables
\usepackage{array} % for better arrays (eg matrices) in maths
\usepackage{paralist} % very flexible & customisable lists (eg. enumerate/itemize, etc.)
\usepackage{verbatim} % adds environment for commenting out blocks of text & for better verbatim
\usepackage{subfig} 
\usepackage{fancyhdr} % This should be set AFTER setting up the page geometry
\pagestyle{fancy} % options: empty , plain , fancy
\renewcommand{\headrulewidth}{0pt} % customise the layout...
\lhead{}\chead{}\rhead{}
\lfoot{}\cfoot{\thepage}\rfoot{}

\usepackage{sectsty}
\allsectionsfont{\sffamily\mdseries\upshape} 
\usepackage[nottoc,notlof,notlot]{tocbibind} % Put the bibliography in the ToC
\usepackage[titles,subfigure]{tocloft} % Alter the style of the Table of Contents
\renewcommand{\cftsecfont}{\rmfamily\mdseries\upshape}
\renewcommand{\cftsecpagefont}{\rmfamily\mdseries\upshape} % No bold!


\begin{document}
\tableofcontents
\newpage
\large

\section{Statistical Quality Control}
\begin{itemize}
\item There is little difference between Statistical Quality Control (SQC) and Statistical Process Control (SPC).  At one time, there might have been some philosophical separation, but today, they exist as general synonyms.  Some prefer SQC because the idea of “quality” is larger and more encompassing than that of “process.”  Others counter this by pointing out that the term “process” is problematic by nature, whereas a focus on “quality” is symptomatic in character.  Still others looked at SQC as the management version of SPC.  The bottom line s simple – both approaches get the job done.

\item As a notable extension of this discussion, we should acknowledge that some quality professionals are still trying to argue that Six Sigma Quality (SSQ) is just another parallax of Total Quality Management (TQM).  To better understand the true parallaxes of Six Sigma, let us consider its three primary domains.  In this manner, we can better understand how the ideas of SQC and SPC support the aims of SSQ.
\end{itemize}


\subsection{Design for Six Sigma (PFSS)}
\begin{itemize}
\item Design for Six Sigma (DFSS) is focused on abating the various forms of risk attributable to the planning of a product, service, system, process, transaction, activity or event – regardless of its nature (industrial or commercial). Looking deeper into DFSS, we discover that it is concerned with two equally important and often interrelated aims.  First, it is concerned with reducing the relative number of risk opportunities and consequential exposures inherent to the functional performance and physical attributes of a design (customer satisfaction issues).  


\item The risk consequence of a design that is subjected to marginal overstress, or the risk of a design feature not having been assigned an adequate performance specification, is examples of such risk.  Second, DFSS is concerned with reducing the relative number of risk opportunities and consequential exposures associated with the “processing viability” of a design (provider satisfaction issues).  The risk of assigning overly conservative tolerances that ultimately result in the need for expensive, higher-grade production processes is an example of such risk.
\end{itemize}

\subsection{Processing for Six Sigma (PFSS)}
\begin{itemize}
\item Processing for Six Sigma (PFSS) is concerned with abating the value-related risk associated with the ongoing operation of systems, processes and supporting activities – regardless of their basic nature (industrial or commercial).  More specifically, PFSS is concerned with reducing the extent of consequence associated with a risk exposure.  In this context, every risk exposure generated by a design has some probability of becoming manifest in the form of a defect, loss, error or quality-related problem during the process of value creation.  From this perspective, it’s easy to see why PFSS is frequently employed for the improvement of process capability, even though it is also employed to improve such other important metrics as cycle time, labor cost, inventory and material cost.
\end{itemize}


\subsection{Managing for Six Sigma (MFSS)}
\begin{itemize}
\item Managing for Six Sigma (MFSS) is the underlying foundation of leadership for a Six Sigma initiative regardless of its nature.  It is concerned with the creation, installation, initialization and utilization of the deployment plans, reporting systems and implementation processes that support DFSS and PFSS.  We can also view MFSS as the unifying leadership component of Six Sigma that overlaps the aims of DFSS and PFSS for the purpose of synergistically realizing value entitlement for the customer and provider in every aspect of the business relationship.
\end{itemize}

\end{document}
