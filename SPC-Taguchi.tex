\documentclass[12pts]{article}

\usepackage{graphicx}

\begin{document}
\tableofcontents

%\section{Statistical Process Control and Industrial Statistics}

\section{Taguchi Method Of Quality Control}


An approach to engineering that emphasizes the roles of research and development, product design and product development in reducing the occurrence of defects and failures in products. The Taguchi method considers design to be more important than the manufacturing process in quality control and tries to eliminate variances in production before they can occur.

Genichi Taguchi, a Japanese engineer and statistician, began formulating the Taguchi Method while developing a telephone-switching system for Electrical Communication Laboratory, a Japanese company, in the 1950s. As a result of his success, he eventually became well-known in both Japan and the United States, with companies such as Toyota, Ford, Boeing and Xerox adopting his methods.

\section{Taguchi Loss Function}
The Taguchi Loss Function is graphical depiction of loss developed by Genichi Taguchi to describe a phenomenon affecting the value of products produced by a company. Praised by Dr. W. Edwards Deming (the business guru of the 1980s American quality movement), it made clear the concept that quality does not suddenly plummet when, for instance, a machinist exceeds a rigid blueprint tolerance. Instead "loss" in value progressively increases as variation increases from the intended condition. This was considered a breakthrough in describing quality, and helped fuel the continuous improvement movement that since has become known as lean manufacturing.




\end{document} 