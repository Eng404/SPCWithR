\subsection{Control Charts}
% http://jjmcd.fedorapeople.org/Download/R/R-2.13-Six_Sigma_with_R_-_A_Tutorial-en-US.pdf

During the Measure phase, one of the first things the Back Belt wants to do is to determine whether the
process is in control with respect to the major 'Y'. The primary tool for this is a control chart. In many
cases, the process may already keep control charts; many do. But there are large number of way in which
control charts are produced, and a great many pitfalls, so the Black Belt would be well advised to examine
the procedures used for the control chart and ensure they are appropriate for his purposes.

The simplest control chart consists of a simple plot of the observed variable versus time, with the control
limits marked on the chart, and sometimes, the specification limit.

The control limits are typically set at +/- three standard deviations. It is important to remember that the
control limits should not be recalculated each time the control chart is redrawn. Rather, they should be set
once, and then changed because of a change in the process.

%-----------------------------------------------------------------------------------%
