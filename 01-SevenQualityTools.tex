
\documentclass[11pt]{article} % use larger type; default would be 10pt

\usepackage[utf8]{inputenc} % set input encoding (not needed with XeLaTeX)

\usepackage{geometry} % to change the page dimensions
\geometry{a4paper}
\usepackage{graphicx} 
\usepackage{booktabs} % for much better looking tables
\usepackage{array} % for better arrays (eg matrices) in maths
\usepackage{paralist} % very flexible & customisable lists (eg. enumerate/itemize, etc.)
\usepackage{verbatim} % adds environment for commenting out blocks of text & for better verbatim
\usepackage{subfig} 
\usepackage{fancyhdr} % This should be set AFTER setting up the page geometry
\pagestyle{fancy} % options: empty , plain , fancy
\renewcommand{\headrulewidth}{0pt} % customise the layout...
\lhead{}\chead{}\rhead{}
\lfoot{}\cfoot{\thepage}\rfoot{}

\usepackage{sectsty}
\allsectionsfont{\sffamily\mdseries\upshape} 
\usepackage[nottoc,notlof,notlot]{tocbibind} % Put the bibliography in the ToC
\usepackage[titles,subfigure]{tocloft} % Alter the style of the Table of Contents
\renewcommand{\cftsecfont}{\rmfamily\mdseries\upshape}
\renewcommand{\cftsecpagefont}{\rmfamily\mdseries\upshape} % No bold!


\begin{document}
\tableofcontents
\newpage
\section{7 Basic Tools of Quality }
\begin{description}
\item[Cause-and-effect diagram]: Identifies many possible causes for an effect or problem and sorts ideas into useful categories.
(also called Ishikawa or fishbone chart)

\item[Check sheet]: A structured, prepared form for collecting and analyzing data; a generic tool that can be adapted for a wide variety of purposes.

\item[Control charts]: Graphs used to study how a process changes over time.

\item[Histogram]: The most commonly used graph for showing frequency distributions, or how often each different value in a set of data occurs.

\item[Pareto chart]: Shows on a bar graph which factors are more significant.

\item[Scatter diagram]: Graphs pairs of numerical data, one variable on each axis, to look for a relationship.
\item[Stratification]: A technique that separates data gathered from a variety of sources so that patterns can be seen (some lists replace “stratification” with “flowchart” or “run chart”).
\end{description}
\subsection{7 Basic Tools of Quality }
{\large
	These are 7 QC tools also known as Ishikawas \textbf{7QC} tools
	\begin{description}
		\item[Cause-and-effect diagram]: Identifies many possible causes for an effect or problem and sorts ideas into useful categories.
		(also called \textit{Ishikawa} or\textit{ fishbone chart})
		
		\item[Check sheet]: A structured, prepared form for collecting and analyzing data; a generic tool that can be adapted for a wide variety of purposes.
		
		\item[Control charts]: Graphs used to study how a process changes over time.
		
		\item[Histogram]: The most commonly used graph for showing frequency distributions, or how often each different value in a set of data occurs.
		
		\item[Pareto chart]: Shows on a bar graph which factors are more significant.
		
		\item[Scatter diagram]: Graphs pairs of numerical data, one variable on each axis, to look for a relationship.
		
		\item[Stratification]: A technique that separates data gathered from a variety of sources so that patterns can be seen (some lists replace “stratification” with “flowchart” or “run chart”).
	\end{description}
	For the sake of brevity, we will only look at a couple of these.
}


\newpage



\section{The \textbf{qcc} R package - The 7QC tools revisited}

\begin{itemize}
	\item The \textbf{qcc} package  was built by Luca Scrucca for nothing but statistical quality control. 
	\item It's extremely easy to use. You provide it with data and it tells you which points are considered to be outliers based on the Shewart Rules. 
	\item It even color codes them based on how irregular each point is. 
	%In the example below you can see that for the last 10 points of the 2nd dataset I shifted he mean of the data from 10 to 11.
	% %
	\item 
	Even though statistical quality control an old topic, statistical quality control is still highly relevant. There are probably have lots of jobs, processes, logs, or databse metric tha could be monitored using control charts.
\end{itemize}


%Yhat Blog


\subsection{qcc : Quality Control Charts}
\textbf{Some Remarks}
\begin{itemize}
	\item Shewhart quality control charts for continuous, attribute and count data.
	\item Cusum and EWMA charts. 
	\item Operating characteristic curves.
	\item Process capability analysis. 
	\item Pareto chart and cause-and-effect chart. 
	\item Multivariate control charts.
\end{itemize}

\subsection{Types of Control Chart supported by qcc}
{
	\large
	\begin{description}
		\item["xbar"]	 	mean -  means of a continuous process variable
		\item["R"]	 	range	 ranges of a continuous process variable
		\item["S"]	 	standard deviation	 standard deviations of a continuous variable
		\item["xbar.one"]	 	mean	 one-at-time data of a continuous process variable
		\item["p"]	 	proportion	 proportion of nonconforming units
		\item["np"]	 	count	 number of nonconforming units
		\item["c"]	 	count	 nonconformities per unit
		\item["u"]	 	count	 average nonconformities per unit
		\item["g"]	 	count	 number of non-events between events
	\end{description}
	
}
\end{document}