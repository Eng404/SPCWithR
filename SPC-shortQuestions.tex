Multiple choice questions

%-----------------------------------------------------------------%
\subsubsection*{Question 1}
1. In the weekly sales in a sports club, we have seen a seasonal pattern (a), exceptional
high values during Christmas week and long weekends(b), and changes of deviation
from seasonal pattern on the week-to-week basis (c). Out of these three [(a), (b),
(c)] sources of variation in the data
A (a) represents an assignable cause of variation, while (b) and (c) represent vari-
ation caused by chance;
B (a) represents variation caused by chance, while (b) and (c) represent assignable
cause of variation;
C all three (a), (b), and (c) represent assignable causes of variation;
D (a) and (b) represent assignable causes of variation, while (b) represents variation
caused by chance;
E all three (a), (b), and (c) represent variation caused by chance.

%-----------------------------------------------------------------%
\subsubsection*{Question 2}
2. The process view of data
A displays data in a time ordered line plot;
B displays data in histogram format;
C processes data by summarising the main features in numerical form and displays
the resulting summaries;
D uses statistical data processing software to provide a professional level summary
of the data;
E summarises the data in the shape of a Normal curve.
%-----------------------------------------------------------------%
\subsubsection*{Question 3}
3. There are available data on weekly precipitation in a town and weekly beer sales in
a local bar over a period of ten years. To see if considering weather prediction in
running the bar business is worth of an effort one could
A draw histograms of these two data sets and examine if their shapes are similar;
B draw a scatterplot view of the data examine its shape and detect any form of
dependence between the variables;
C check during which week the most beer was sold and check how much rain there
was during this week;
D write data into the table and examine if there are any relation between values;
E check if the histograms for both data sets resemble a Normal curve.
%-----------------------------------------------------------------%
\subsubsection*{Question 4}
4. Assignable causes of variation
A always follow the Normal model;
B follow the Normal model when there are no chance causes of variation;
C correspond to the most frequent classes in a histogram;
D are the many unpredictable but collectively influential factors that affect a pro-
cess or system;
E are the few factors, individually influential and with predictable effect, that affect
a process or system.

%-----------------------------------------------------------------%
\subsubsection*{Question 5}
5. The Normal model for chance variation
A is a flexible family of frequency distributions with mean and standard deviation
as parameters;
B is the usual model for the linear part of a simple linear regression equation;
C has a frequency distribution which is evenly spread across the scale;
D is the standard model for the assignable causes in a chance system;
E is the result of a series of orderly experiments.


%===================================================%

Multiple choice questions
1. The central horizontal line on the control charts
A is computed as the overall average all available data;
B is computed by taking the midpoint between maximal and minimal value in the
data;
C is not affected by the values in the data that are close to it;
D is computed by eliminating the effect of unusually large or small values;
E is evaluated by properly scaling the standard deviation.
2. The upper and lower control limits
A are approximately set up so that more than 99% data falls in between them if
process is in control;
B are approximately set up so that more than 99% data falls in between them
irrespectively if process is or is not in control;
C are based only on the value of the overall mean;
D do not vary for different subsample sizes in ¯X charts;
E are obtained by using the largest and the smallest data points.
%-----------------------------------------------------------------%
\subsubsection*{Question 3}
3. The occurrence of an ’out-of-control’ point on a control chart
A is said to be a false alarm if it not as extreme as other cases of ’out-of-control’
occurences;
B is not likely to happend so is considered statistically significant for detecting an
assignable cause of variation;
C is said to be statistically significant only if the process was not properly centred
in the first place;
D is less likely when using ”2- limits” than when using ”3- limits”;
E is less likely when the process is off centre than when it is properly centred.


%-----------------------------------------------------------------%
\subsubsection*{Question 4}
4. When using numbers of defective items in subgroups sampled from a process as the
basis for a control chart, which one of the following is correct?
A np charts are so called because the false alarm rates are based on normal prob-
abilities, even though the Normal model is not strictly correct;
B np charts are so called because the expected number of defects is n times the
proportion of defects characteristic of the process;
C the lower control limit in an np chart calculated from the standard formula can
never be less than 0;
D it is not possible to get a value below the lower control limit when using an np
chart because the lower control limit cannot be less than 0;
E it is not possible to calculate an accurate standard error formula for use in an np
chart because there is no  involved.
Problems
1. A certain process is observed and recorded daily.
• How improbable is a point outside the 3 limits when the process is in control?
• How often will such a point occur when observed daily (find expected frequency
in terms of days)?
• How improbable is a point outside the 2 limits when the process is in control?
• How often will such a point occur?
2. Suppose that for the above process ¯X and R control charts have been created based
on the subsample means of weekly observation (thus the subsample size is n = 7).
• Using the AIAG chart that is presented in Figure 4.6 of the textbook, identify
the values of A2, D3 and D4 for this process (see also our lecture slides where
these values have been presented).
• After analysing twelve month data it have been found out that the process has
been in control for the entire year and the average value of the range has been
found to be 6.3 while the mean value was 78. Find the control limits for the
¯X
and R charts and present them on a graph for a weekly control chart.
• From the obtained data calculate the standard deviation for this process.
• Using the obtained values compute the percentage of the items that will not
fell within the ¯X control belt.
3. Assuming a value of 7.3 [mm] for , use the Normal table to predict the proportion
of clips whose gaps fail to meet the specification limits of 50[mm] to 90[mm]
• when the process mean is 74[mm],
• when the process mean is 67[mm].
