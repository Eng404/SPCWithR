\documentclass[12pt]{article}

\usepackage{framed}
\usepackage{amsmath}
\usepackage{amssymb}
\usepackage{graphics}
\usepackage{graphicx}
%opening
\title{Statistical Process Control}
%\author{MA4605}

\begin{document}
Statistical Process Control (SPC)  
Statistical Process Control (SPC) is an industry-standard methodology for measuring and controlling quality during the manufacturing process. Quality data in the form of Product or Process measurements are obtained in real-time during manufacturing. 
This data is then plotted on a graph with pre-determined control limits. Control limits are determined by the capability of the process, whereasspecification limits are determined by the client's needs.

%==========================%
Control Limits on an XBar Range Chart
Data that falls within the control limits indicates that everything is operating as expected. Any variation within the control limits is likely due to a common cause—the natural variation that is expected as part of the process. 
If data falls outside of the control limits, this indicates that an assignable cause is likely the source of the product variation, and something within the process should be changed to fix the issue before defects occur.


\section*{Statistical Process Control}
	
	Statistical process control (SPC) is a method of quality control which uses statistical methods. SPC is applied in order to monitor and control a process. Monitoring and controlling the process ensures that it operates at its full potential. At its full potential, the process can make as much conforming product as possible with a minimum (if not an elimination) of waste (rework or Scrap).
	
	SPC can be applied to any process where the "conforming product" (product meeting specifications) output can be measured. Key tools used in SPC include control charts; a focus on continuous improvement; and the design of experiments. An example of a process where SPC is applied is manufacturing lines.
	

	
	Commonly used in the manufacturing process, statistical process control (SPC) makes use of statistical facts gleaned through statistical analysis to both monitor and control virtually any process where output can be measured. SPC makes use of a variety of tools inherent to the method to include experimentation, control charts and continuous improvement processes.
	
	The key difference between SPC and other process control methods is a focus on quantitative analysis, rather than opinion, when analyzing variations in a process. Applied to a wide range of processes aside from manufacturing, statistical process control focuses on identifying sources of variation and determining the extent of that variation. Based on such information, managers can make decisions regarding whether the variation is acceptable, if it signifies a problem or a positive causation that needs replicating.
\section*{Introductory Definitions}
\begin{itemize}
\item Statistical process control refers to the application of the methods of statistical quality control to the
monitoring of processes (and not just, as in the earlier practice, to the inspection of the final outputs of the
processes).

\item The purpose is to control the quality of product or service outputs from a process by maintaining
control of the process. When a process is described as being \textbf{in control}, it means that the amount of variation
in the output is relatively constant and within established limits that are deemed acceptable. There are two kinds
of causes of variation in a process.

\item \textbf{Common causes}, or chance causes, of variation are due to factors that are
inherent in the design of the system, and reflect the usual amount of variation to be expected.
\item \textbf{Assignable causes},
or special causes, of variation are due to unusual factors that are not part of the process design and not ordinarily
part of the process.

\item A \textbf{stable process} is one in which only common causes of variation affect the output quality. Such a process
can also be described as being in a state of statistical control.

\item An \textbf{unstable} process is one in which both
assignable causes and common causes affect the output quality. (Note that, by definition, the common causes
are always present.) Such a process can also be described as being out of control, particularly when the
assignable cause is controllable.
\end{itemize}

\newpage

\section*{Short Theory Questions}

\begin{itemize}

\item \textbf{What is a process?}\\
It is a sequence of operations by which such inputs as labor, materials, and methods are transformed into
either product or service outputs.

\item \textbf{Differentiate common (or chance) causes of variation in the quality of process output from assignable
(or special) causes.}\\
Common causes are inherent in the design of the system and reflect the typical variation to be expected.
Assignable causes are special causes of variation that are ordinarily not part of the process, and should be corrected
as warranted.

\item \textbf{Differentiate a stable process from an unstable process.}\\
A stable process is one that exhibits only common-cause variation. An unstable process exhibits variation due
to both assignable and common causes.

\item \textbf{Describe how the output of a stable process can be improved. What actions do not improve a stable
process, but rather, make the output more variable?}\\
A stable process can be improved only by changing the design of the process. Attempts to make adjustments to
a stable process, which is called tampering, results in more variation in the quality of the output.

\item \textbf{What is the purpose of maintaining control charts? }\\
Control charts are used to detect the occurrence of assignable causes affecting the quality of process output.




\item \textbf{When is a process considered to be “out of control,” and how can it be brought under control?}\\
A process is “out of control,” or unstable, when there is assignable-cause variation in the quality of the
output. Improvement can be achieved by identifying and removing the assignable cause(s).

\item\textbf{ What is tampering in the context of process control?}\\
It is the attempt to adjust a process that is in fact stable and includes only common-cause variation.

\item \textbf{As contrasted to critical values being set according to levels of significance in hypothesis testing, what is
the standard used for setting the control limits in process control?}\\
The standard practice is to set the control limits at three standard errors from the centerline. Thus, they are
called 3-sigma limits.


\item \textbf{Other than applying the 3-sigma rule for detecting the presence of an assignable cause, what else do we look for
when studying a control chart?} \\
The existence of a pattern in the sequence of sample results. (See Nelson's Eight Tests)
\end{itemize}
\end{document} 
