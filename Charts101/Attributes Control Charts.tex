
6.3.3.

What are Attributes Control Charts?

Attributes data arise when classifying or counting observations	The Shewhart control chart plots quality characteristics that can be measured and expressed numerically. We measure weight, height, position, thickness, etc. If we cannot represent a particular quality characteristic numerically, or if it is impractical to do so, we then often resort to using a quality characteristic to sort or classify an item that is inspected into one of two "buckets".
An example of a common quality characteristic classification would be designating units as "conforming units" or "nonconforming units". Another quality characteristic criteria would be sorting units into "non defective" and "defective" categories. Quality characteristics of that type are called attributes.

Note that there is a difference between "nonconforming to an engineering specification" and "defective" -- a nonconforming unit may function just fine and be, in fact, not defective at all, while a part can be "in spec" and not fucntion as desired (i.e., be defective).

Examples of quality characteristics that are attributes are the number of failures in a production run, the proportion of malfunctioning wafers in a lot, the number of people eating in the cafeteria on a given day, etc.

Types of attribute control charts	Control charts dealing with the number of defects or nonconformities are called c charts (for count).
Control charts dealing with the proportion or fraction of defective product are called  p charts (for proportion).

There is another chart which handles defects per unit, called the u chart (for unit). This applies when we wish to work with the average number of nonconformities per unit of product.

For additional references, see Woodall (1997) which reviews papers showing examples of attribute control charting, including examples from semiconductor manufacturing such as those examining the spatial depencence of defects.

