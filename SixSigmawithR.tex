\documentclass[]{article}

%opening


\begin{document}
\tableofcontents
%------------------------------------------------------- %

\section{Industrial Statistics with \texttt{R}}
\begin{quote}
Six Sigma has arisen in the last two decades as a breakthrough Quality Management Methodology. With Six Sigma, we are solving problems and improving processes using as a basis one of the most powerful tools of human development: the scientific method. For the analysis of data, Six Sigma requires the use of statistical software, being R an Open Source option that fulfills this requirement. R is a software system that includes a programming language widely used in academic and research departments. Nowadays, it is becoming a real alternative within corporate environments.

 

The aim of this book is to show how R can be used as the software tool in the development of Six Sigma projects. The book includes a gentle introduction to Six Sigma and a variety of examples showing how to use R within real situations. It has been conceived as a self contained piece. Therefore, it is addressed not only to Six Sigma practitioners, but also to professionals trying to initiate themselves in this management methodology. The book may be used as a text book as well.
\end{quote}
%------------------------------------------------------- %
\subsection{Resources and R packages}
\begin{itemize}
\item \textit{\textbf{SixSigma}} \texttt{R} package
\item \textit{Multivariate Statistical Quality Control Using R - Springer}
\end{itemize}
%-------------------------------------------------------- %
\newpagee
\subsection{\textbf{SixSigma}: Six Sigma Tools for Quality and Process Improvement}

\begin{itemize}
\item This package contains functions and utilities to perform Statistical Analyses in the Six Sigma way. 

\item Through the DMAIC cycle (Define, Measure, Analyze, Improve, Control), you can manage several Quality Management studies: Gage R\&R, Capability Analysis, Control Charts, Loss Function Analysis, etc. 

\item Data frames used in the book "\textit{Six Sigma with R}" (Springer, 2012) are also included in the package.
\end{itemize}
%------------------------------------------------------ %



\subsection{\textbf{spc}: Statistical Process Control – Collection of Some Useful Functions}

Evaluation of control charts by means of the zero-state, steady-state ARL (Average Run Length) and RL quantiles. Setting up control charts for given in-control ARL. 

The control charts under consideration are one- and two-sided EWMA, CUSUM, and Shiryaev-Roberts schemes for monitoring the mean of normally distributed independent data. ARL calculation of the same set of schemes under drift are added. Other charts and parameters are in preparation. Further SPC areas will be covered as well (sampling plans, capability indices ...).

\subsection{\textbf{spcadjust}: Functions for calibrating control charts}

This package calibrates thresholds of control charts such as CUSUM charts based on past data, taking estimation error into account.

\subsection{ \textbf{IQCC}: Improved Quality Control Charts}

Builds statistical control charts with exact limits for univariate and multivariate cases.
\subsection{Loss Function Analysis with \texttt{R}}
\end{document}