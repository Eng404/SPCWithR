\documentclass[12pt]{article}
\usepackage{framed}
\usepackage{amsmath}
\usepackage{amssymb}
\usepackage{graphics}
\begin{document}
\tableofcontents
\section{Process Capability}


Process capability compares the output of an in-control process to the specification limits by using capability indices. The comparison is made by forming the ratio of the spread between the process specifications (the specification "width") to the spread of the process values, as measured by 6 process standard deviation units (the process "width").
\section{What is Process Capability?}

Process capability compares the output of an in-control process to the specification limits by using capability indices. The comparison is made by forming the ratio of the spread between the process specifications (the specification "width") to the spread of the process values, as measured by 6 process standard deviation units (the process "width").

\subsection{Process Capability Index}

\textit{\textbf{Process Capability Index}} is used to find out how well the process is centered within the specification limits.It is denoted by Cpk.

\[Cpk = Cp(1-K)\]
\[K = \frac{2(\mbox{Design Target} - \mbox{Process Average})}{ (USL - LSL)}
\]
Where,

\begin{itemize}
\item Cp = Process Capability
\end{itemize}

Design target is the actual specification targetted without +/- allowance.


\end{document}