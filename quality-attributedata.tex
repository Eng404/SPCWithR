% !TEX TS-program = pdflatex
% !TEX encoding = UTF-8 Unicode

% This is a simple template for a LaTeX document using the "article" class.
% See "book", "report", "letter" for other types of document.

\documentclass[11pt]{article} % use larger type; default would be 10pt

\usepackage[utf8]{inputenc} % set input encoding (not needed with XeLaTeX)

%%% Examples of Article customizations
% These packages are optional, depending whether you want the features they provide.
% See the LaTeX Companion or other references for full information.

%%% PAGE DIMENSIONS
\usepackage{geometry} % to change the page dimensions
\geometry{a4paper} % or letterpaper (US) or a5paper or....
% \geometry{margin=2in} % for example, change the margins to 2 inches all round
% \geometry{landscape} % set up the page for landscape
%   read geometry.pdf for detailed page layout information

\usepackage{graphicx} % support the \includegraphics command and options

% \usepackage[parfill]{parskip} % Activate to begin paragraphs with an empty line rather than an indent

%%% PACKAGES
\usepackage{booktabs} % for much better looking tables
\usepackage{array} % for better arrays (eg matrices) in maths
\usepackage{paralist} % very flexible & customisable lists (eg. enumerate/itemize, etc.)
\usepackage{verbatim} % adds environment for commenting out blocks of text & for better verbatim
\usepackage{subfig} % make it possible to include more than one captioned figure/table in a single float
% These packages are all incorporated in the memoir class to one degree or another...

%%% HEADERS & FOOTERS
\usepackage{fancyhdr} % This should be set AFTER setting up the page geometry
\pagestyle{fancy} % options: empty , plain , fancy
\renewcommand{\headrulewidth}{0pt} % customise the layout...
\lhead{}\chead{}\rhead{}
\lfoot{}\cfoot{\thepage}\rfoot{}

%%% SECTION TITLE APPEARANCE
\usepackage{sectsty}
\allsectionsfont{\sffamily\mdseries\upshape} % (See the fntguide.pdf for font help)
% (This matches ConTeXt defaults)

%%% ToC (table of contents) APPEARANCE
\usepackage[nottoc,notlof,notlot]{tocbibind} % Put the bibliography in the ToC
\usepackage[titles,subfigure]{tocloft} % Alter the style of the Table of Contents
\renewcommand{\cftsecfont}{\rmfamily\mdseries\upshape}
\renewcommand{\cftsecpagefont}{\rmfamily\mdseries\upshape} % No bold!


\begin{document}

\section{What are Attributes Control Charts?}

\textbf{Attributes data arise when classifying or counting observations.}  

The Shewhart control chart plots quality characteristics that can be measured and expressed numerically. We measure weight, height, position, thickness, etc. If we cannot represent a particular quality characteristic numerically, or if it is impractical to do so, we then often resort to using a quality characteristic to sort or classify an item that is inspected into one of two "buckets".


An example of a common quality characteristic classification would be designating units as "conforming units" or "nonconforming units". Another quality characteristic criteria would be sorting units into "non defective" and "defective" categories. Quality characteristics of that type are called \textbf{attributes}.


Note that there is a difference between "nonconforming to an engineering specification" and "defective" -- a nonconforming unit may function just fine and be, in fact, not defective at all, while a part can be "in spec" and not function as desired (i.e., be defective).
\subsection{Examples}
Examples of quality characteristics that are attributes are the number of failures in a production run, the proportion of malfunctioning wafers in a lot, the number of people eating in the cafeteria on a given day, etc.

\subsection{Types of attribute control charts}	 
\begin{itemize}
\item Control charts dealing with the number of defects or nonconformities are called c charts (for count).
\item Control charts dealing with the proportion or fraction of defective product are called  p charts (for proportion).
\item 
There is another chart which handles defects per unit, called the \textbf{u chart} (for unit). This applies when we wish to work with the average number of nonconformities per unit of product.
\end{itemize}



\end{document}
