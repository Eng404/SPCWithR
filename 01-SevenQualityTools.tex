
\documentclass[11pt]{article} % use larger type; default would be 10pt

\usepackage[utf8]{inputenc} % set input encoding (not needed with XeLaTeX)

\usepackage{geometry} % to change the page dimensions
\geometry{a4paper}
\usepackage{graphicx} 
\usepackage{booktabs} % for much better looking tables
\usepackage{array} % for better arrays (eg matrices) in maths
\usepackage{paralist} % very flexible & customisable lists (eg. enumerate/itemize, etc.)
\usepackage{verbatim} % adds environment for commenting out blocks of text & for better verbatim
\usepackage{subfig} 
\usepackage{fancyhdr} % This should be set AFTER setting up the page geometry
\pagestyle{fancy} % options: empty , plain , fancy
\renewcommand{\headrulewidth}{0pt} % customise the layout...
\lhead{}\chead{}\rhead{}
\lfoot{}\cfoot{\thepage}\rfoot{}

\usepackage{sectsty}
\allsectionsfont{\sffamily\mdseries\upshape} 
\usepackage[nottoc,notlof,notlot]{tocbibind} % Put the bibliography in the ToC
\usepackage[titles,subfigure]{tocloft} % Alter the style of the Table of Contents
\renewcommand{\cftsecfont}{\rmfamily\mdseries\upshape}
\renewcommand{\cftsecpagefont}{\rmfamily\mdseries\upshape} % No bold!


\begin{document}
\tableofcontents
\newpage
\section{7 Basic Tools of Quality }
\begin{description}
\item[Cause-and-effect diagram]: Identifies many possible causes for an effect or problem and sorts ideas into useful categories.
(also called Ishikawa or fishbone chart)

\item[Check sheet]: A structured, prepared form for collecting and analyzing data; a generic tool that can be adapted for a wide variety of purposes.

\item[Control charts]: Graphs used to study how a process changes over time.

\item[Histogram]: The most commonly used graph for showing frequency distributions, or how often each different value in a set of data occurs.

\item[Pareto chart]: Shows on a bar graph which factors are more significant.

\item[Scatter diagram]: Graphs pairs of numerical data, one variable on each axis, to look for a relationship.
\item[Stratification]: A technique that separates data gathered from a variety of sources so that patterns can be seen (some lists replace “stratification” with “flowchart” or “run chart”).
\end{description}
\end{document}