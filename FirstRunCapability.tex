
\documentclass[11pt]{article} % use larger type; default would be 10pt

\usepackage[utf8]{inputenc} % set input encoding (not needed with XeLaTeX)

\usepackage{geometry} % to change the page dimensions
\geometry{a4paper}
\usepackage{graphicx} 
\usepackage{booktabs} % for much better looking tables
\usepackage{array} % for better arrays (eg matrices) in maths
\usepackage{paralist} % very flexible & customisable lists (eg. enumerate/itemize, etc.)
\usepackage{verbatim} % adds environment for commenting out blocks of text & for better verbatim
\usepackage{subfig} 
\usepackage{fancyhdr} % This should be set AFTER setting up the page geometry
\pagestyle{fancy} % options: empty , plain , fancy
\renewcommand{\headrulewidth}{0pt} % customise the layout...
\lhead{}\chead{}\rhead{}
\lfoot{}\cfoot{\thepage}\rfoot{}

\usepackage{sectsty}
\allsectionsfont{\sffamily\mdseries\upshape} 
\usepackage[nottoc,notlof,notlot]{tocbibind} % Put the bibliography in the ToC
\usepackage[titles,subfigure]{tocloft} % Alter the style of the Table of Contents
\renewcommand{\cftsecfont}{\rmfamily\mdseries\upshape}
\renewcommand{\cftsecpagefont}{\rmfamily\mdseries\upshape} % No bold!


\begin{document}

\section{First Run Capability}
\begin{itemize}
\item "First run capability" is a relatively simple concept with a direct method of calculation. The concept is reflected in the statement below:

\item "If you loaded the process inputs with enough raw material and parts to produce, say, 10,000 parts, how many would be produced if there were no repairs or other corrections to the process?"

\item If 9,000 parts were produced, your first run capability would be 90\%. The objective then is to track first run capability and use the data to improve the process.

\item To further illustrate the concept suppose there was a 3 step process. Each step in the process passed 90\% of the work presented to it. Then your first run capability would be 72\%. (0.9 x 0.9 x 0.9). The 28\% that is lost represents waste and/or added cost to repair. There are added quality concerns about the 28\% the deviate from the straight-through process.
\end{itemize}
\end{document}