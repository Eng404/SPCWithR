\documentclass[12pts]{article}

\usepackage{graphicx}

\begin{document}
\tableofcontents

%\section{Statistical Process Control and Industrial Statistics}

\section{Acceptance Sampling}

A statistical measure used in quality control. A company cannot test every one of its products due to either ruining the products, or the volume of products being too large. Acceptance sampling solves this by testing a sample of product for defects. The process involves batch size, sample size and the number of defects acceptable in the batch. This process allows a company to measure the quality of a batch with a specified degree of statistical certainty without having to test every unit of product. The statistical reliability of a sample is generally measured by a t-statistic.

Probability is a key factor in acceptance sampling, but it is not the only factor. If a company makes a million products and tests 10 units with one default, an assumption would be made on probability that 100,000 of the 1,000,000 are defective. However, this could be a grossly inaccurate representation. More reliable conclusions can be made by increasing the batch size higher than 10, and increasing the sample size by doing more than just one test and averaging the results. When done correctly, acceptance sampling is a very effective tool in quality control.

\section{Acceptance Testing}

The acceptance testing process acts as a form of quality control to identify problems and defects at a stage where any issues can still be corrected relatively painlessly. The term “acceptance testing” is commonly used in the field of engineering, particularly in reference to software testing or mechanical hardware testing.

\section{Acceptable Quality Level (AQL)}

A statistical measurement of the maximum number of defective goods considered acceptable in a particular sample size. If the acceptable quality level (AQL) is not reached for a particular sampling of goods, manufacturers will review the various parameters in the production process to determine the areas causing the defects.

The AQL is an important statistic to companies seeking a Six Sigma level of quality control.

The AQL of a product can vary from industry to industry. For example, medical products are more likely to have a more stringent AQL because defective products can result in health risks. Companies have to weigh the added cost associated with the stringent testing and potentially higher spoilage due to a lower defect acceptance with the potential cost of a product recall.


\end{document}
